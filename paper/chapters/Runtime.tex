%!TEX root = ../paper.tex
\chapter{程序运行效果}
	\section{星座图}
		\par 在电脑上,如果程序正常运行,则可以在接收端看到如图\ref{fig:dvbt_rx_map_1}和\ref{fig:dvbt_rx_map_2}星座图。
		\begin{figure}[htp]
			\centering
			\subfigure[DVB-T接收端星座图]{%
				\includegraphics[width=0.4\textwidth]{figures/dvbt_rx_map_1.png}
				\label{fig:dvbt_rx_map_1}}
			\subfigure[DVB-T接收端星座图]{%
				\includegraphics[width=0.4\textwidth]{figures/dvbt_rx_map_2.png}
				\label{fig:dvbt_rx_map_2}} \\
		\end{figure}
	\section{频谱}
		\par 有如下几种方式查看输出频谱:
		\par $\bullet$ 使用GNU Radio构建一个简易的频谱仪。
		\par 按图搭建框图\ref{fig:dvbt_rx_fft}即可,其效果如图\ref{fig:dvbt_rx_fft_1}所示。
		\begin{figure}[htp]
			\centering
			\includegraphics[width=13cm]{figures/dvbt_rx_fft.png}
			\caption{GNU Radio频谱仪}
			\label{fig:dvbt_rx_fft}
		\end{figure}
		\begin{figure}[htp]
			\centering
			\includegraphics[width=13cm]{figures/dvbt_rx_fft_1.png}
			\caption{GNU Radio频谱仪}
			\label{fig:dvbt_rx_fft_1}
		\end{figure}
		\par $\bullet$ 使用频谱仪
		\par 带宽为2MHz时频谱如图\ref{fig:dvbt_BW_2MHz}所示,带宽为5MHz时频谱如图\ref{fig:dvbt_BW_5MHz}所示。
		\begin{figure}[htp]
			\centering
			\subfigure[带宽为2MHz时频谱]{%
				\includegraphics[width=0.4\textwidth]{figures/dvbt_BW_2MHz.jpg}
				\label{fig:dvbt_BW_2MHz}}
			\subfigure[带宽为5MHz时频谱]{%
				\includegraphics[width=0.4\textwidth]{figures/dvbt_BW_5MHz.jpg}
				\label{fig:dvbt_BW_5MHz}} \\
		\end{figure}
		\par $\bullet$ 使用gqrx
		\par 直接在gqrx中调节到相应频段查看即可,不再赘述。
	\section{距离}
		\par 收发端之间距离较近时可以在当前目录下看到生成的\lstinline[language=sh]{test_out.ts}文件,使用VLC或者Mplayer可以播放出视频,如图\ref{fig:dvbt_rx_TS}。
		\begin{figure}[htp]
			\centering
			\includegraphics[width=13cm]{figures/dvbt_rx_TS.png}
			\caption{视频输出}
			\label{fig:dvbt_rx_TS}
		\end{figure}
		\par 由于USRP的输出功率,信号能传播大约一间屋子的距离(约5米),超过这段距离将会出现以如图\ref{fig:dvbt_rx_2MHz_ranged_2}无法解星座映射以及如图\ref{fig:dvbt_rx_TS_broken}视频播放失真的情况。
		\begin{figure}[htp]
			\centering
			\subfigure[约5m处星座图]{%
				\includegraphics[width=0.4\textwidth]{figures/dvbt_rx_2MHz_ranged_2.png}
				\label{fig:dvbt_rx_2MHz_ranged_2}}
			\subfigure[约5m处视频解码]{%
				\includegraphics[width=0.4\textwidth]{figures/dvbt_rx_TS_broken.png}
				\label{fig:dvbt_rx_TS_broken}} \\
		\end{figure}
	\section{树莓派发射}
		\par 将设备按图\ref{fig:dvbt_raspi}所示连接,便可以在接收端上看到\lstinline[language=sh]{test_out.ts}但是由于带宽太小,误码率高,在OFDM接收端处无法解码,该文件无法流畅播放。
		\par 由于树莓派的计算性能与IO性能较弱,仅能提供约500KHz的带宽,如图\ref{fig:dvbt_raspi_bandwidth},也许将部分模块交由FPGA处理会有巨大的提升,或者通过gpu来计算相关模块,有待后续开发者继续研究。
		\begin{figure}[htp]
			\centering
			\subfigure[树莓派连接图]{%
				\includegraphics[width=0.4\textwidth]{figures/dvbt_raspi.jpg}
				\label{fig:dvbt_raspi}}
			\subfigure[树莓派运行频谱]{%
				\includegraphics[width=0.4\textwidth]{figures/dvbt_raspi_bandwidth.jpg}
				\label{fig:dvbt_raspi_bandwidth}} \\
		\end{figure}
