%!TEX root = ../paper.tex
\chapter{总结与展望}
	\section{总结}
	\par 本文从树莓派下Linux环境的安装开始,介绍了GNU Radio相关的环境配置,着重介绍了CMake、git和GNU Radio的相关知识,找到了一种在arm架构、低版本GNU Radio下编译gr-dvbt的方法。之后简要介绍了DVB-T的相关知识,介绍了DVB-T发射系统中各个模块的作用,介绍了一些模块的具体实现方法,并给出了一些模块的源码。接下来给出了一个通过GNU Radio搭建一个DVB-T收发系统的框图,该系统能够很好进行MPEG-2 TS视频流的收发,可能由于系统的性能,暂时还没有实现视频的实时收发。在具体实现的后面也给出了该系统的运行效果,包括星座图以及示波器波形,并发现该套系统的最大收发距离约为5米。整个系统在i5-3337U平台上运行时带宽最高可以达到5MHz,搭载在树莓派上时由于系统的处理性能问题,整个系统的带宽仅能达到500KHz,无法完成视频接收端的解码。使用USRP时发现带宽与采样率相同,但并未找到其相同的准确原因,有待更加深入的研究。
	\section{展望}
	\par 本设计通过GNU Radio实现,由于GNU Radio是开源软件,可以通过修改其相关代码来实现一些特殊的功能。可以将相关运算移动到FPGA上来加速运算,因为树莓派提供了一个比较羸弱的GPU,也许可以通过GPU计算FFT来加速整个程序的运行。DVB-T协议对多普勒频移比较敏感,可以采用DVB-H协议来搭建整个系统实现数据信息在移动端的收发。可以修改能量扩散的模块来传递自己需要的格式,并将RS编码等编码方式更换为特定的编码来实现传递其他类型数据信息的系统。由于在开发中因为需要的依赖问题,有可能需要进行频繁的系统重做之类的操作,可以将整个系统放置到Docker上来实现整个系统的版本控制与回溯,从而缩短开发过程。后续开发者可以参考本文来进行快速搭建与修改。