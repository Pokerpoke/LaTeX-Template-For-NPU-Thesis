%!TEX root = ../paper.tex
\renewcommand{\baselinestretch}{1.5}
\fontsize{12pt}{13pt}\selectfont
\phantomsection
\chapter*{绪~~~~论}
\addcontentsline{toc}{chapter}{\fHei 绪论}

\par 本文第一章将简要介绍树莓派的相关知识,给出树莓派烧录系统的一般步骤,完成整个系统最基础的配置,接下来所有的操作都将基于Linux。
\par 第二章将详细介绍在树莓派平台上安装GNU Radio的方法,着重介绍GNU Radio程序编写过程中使用的一些库和软件,如果后续开发者需要进行源码的修改,这些知识必不可少,尤其是CMake与git,熟悉这些软件的使用能给开发带来极大的便利,也能为开发其他开源项目提供便利。
\par 第三章将简要说明DVB-T系统的结构与原理,通过框图解释了整个系统的运行过程,方便读者对整个系统有个总体的了解。
\par 第四章将重点介绍DVB-T发射端的原理,相关模块的参数,同时给出相关的源码,方便读者理解代码逻辑。
\par 第五章给出了DVB-T在GNU Radio上的实现方案,及设备连接示意图,按图连接,设置参数即可运行。
\par 第六章将通过星座图、频谱等展示整个系统的性能,也会指出整个系统的不足、可供改进之处。
\par 第七章将总结整个文章,做出评价与展望,给后续开发者提供一些可行的研究方向。

\clearpage