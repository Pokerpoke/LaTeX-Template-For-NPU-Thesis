%!TEX root = ../paper.tex
\chapter{DVB-T系统概述}
	\section{DVB-T系统结构与原理}
	\label{sec:dvbt_summary}
		\par 地面数字视频广播(英语:Digital Video Broadcasting-Terrestrial,缩写:DVB-T),是欧洲广播联盟在1997年发布的数字地面电视视频广播传输,最早是1998年在英国实行广播\ucite{wiki:DVB-T}。
		\par DVB-T是利用开路地面传输媒介进行MPEG-2数字电视的传输标准。由于地面电视的特殊环境,参与DVB组织的专家们选定了码分正交频分复用(COFDM)的信道调制技术,并使用强大的纠错码,达到频谱利用效率与传输可靠性的平衡。COFDM提供了两种子载波数量(2k模式和8k模式),3种调制方式(QPSK,16QAM和64QAM),4种保护间隔(1/4,1/8,1/6和1/32),并支持小范围和大范围的单频网运行(SFN),系统支持目前模拟电视系统的8MHz带宽,7MHz和6MHz带宽,系统还支持等级调制\ucite{what_is_dvb_t}。
		\par 图\ref{fig:dvbt_tx}描述了DVB-T发射端系统。
		\begin{figure}[htb]
	\centering
	\begin{tikzpicture}[node distance=3em]
		\node(MPEG_2_stream)[process_vertical]{\rotatebox{-90}{MPEG-2 TS}\\流};
		\node(energy_dispersal)[process_vertical,right of=MPEG_2_stream]{能量扩散};
		\node(RS_encoder)[process_vertical,right of=energy_dispersal]{R\\S编码};
		\node(convolutional_interleaver)[process_vertical,right of=RS_encoder]{卷积内交织};
		\node(inner_coder)[process_vertical,right of=convolutional_interleaver]{内编码};
		\node(bit_inner_interleaver)[process_vertical,right of=inner_coder]{比特内交织};
		\node(symbol_inner_interleaver)[process_vertical,right of=bit_inner_interleaver]{符号内交织};
		\node(dvbt_map)[process_vertical,below=2em of symbol_inner_interleaver]{星座映射};
		\node(reference_signal)[process_vertical,left of=dvbt_map]{参考信号};
		\node(IFFT)[process_vertical,left of=reference_signal]{反傅里叶变换};
		\node(OFDM_CP)[process_vertical,left of=IFFT]{\rotatebox{-90}{OFDM}\\循环前缀};
		\node(rational_resampler)[process_vertical,left of=OFDM_CP]{重采样};
		\node(USRP_sink)[process_vertical,left of=rational_resampler]{\rotatebox{-90}{USRP}\\发射};
				
		\draw[arrow](MPEG_2_stream)--(energy_dispersal);
		\draw[arrow](energy_dispersal)--(RS_encoder);
		\draw[arrow](RS_encoder)--(convolutional_interleaver);
		\draw[arrow](convolutional_interleaver)--(inner_coder);
		\draw[arrow](inner_coder)--(bit_inner_interleaver);
		\draw[arrow](bit_inner_interleaver)--(symbol_inner_interleaver);
		\draw[arrow](symbol_inner_interleaver)--(dvbt_map);
		\draw[arrow](dvbt_map)--(reference_signal);
		\draw[arrow](reference_signal)--(IFFT);
		\draw[arrow](IFFT)--(OFDM_CP);
		\draw[arrow](OFDM_CP)--(rational_resampler);
		\draw[arrow](rational_resampler)--(USRP_sink);
	\end{tikzpicture}
	\caption{DVB-T发射端系统框图}
	\label{fig:dvbt_tx}
\end{figure}
\endinput
		\par 图\ref{fig:dvbt_rx}描述了DVB-T接收端系统。
		\begin{figure}[thb]
	\begin{tikzpicture}
	\end{tikzpicture}
\end{figure}
\endinput
		\par DVB-T系统中可以调节的参数见表\ref{table:params_of_dvbt}。
		\begin{table}[!htbp]
			\centering
			\caption{DVB-T供调节参数}
			\begin{tabular}{|l|p{0.6\columnwidth}|}
				\hline\hline
				内纠错码率(FEC) & 1/2 , 2/3 , 3/4 , 5/6 , 7/8                            \\
				\hline
				子载波调制方式    & QPSK , 16QAM , 64QAM                                   \\
				\hline
				保护间隔             & 1/4 , 1/8 , 1/16 , 1/32                                \\
				\hline
				等级调制参数       & $\alpha$=1(等级) , $\alpha$=2 , 4(非等级) \\
				\hline
				载波数量             & 2k : 1705 , 8k : 6817                                  \\
				\hline\hline
			\end{tabular}
			\label{table:params_of_dvbt}
		\end{table}
		\endinput