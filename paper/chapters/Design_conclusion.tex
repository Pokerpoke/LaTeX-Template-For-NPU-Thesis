%!TEX root = ../paper.tex
\renewcommand{\baselinestretch}{1.5}
\fontsize{12pt}{13pt}\selectfont
\phantomsection
\chapter*{毕业设计小结}
\addcontentsline{toc}{chapter}{\fHei 毕业设计小结}

\par 本文第一章简要介绍了树莓派的相关知识,给出了给树莓派烧录系统的一般步骤,完成了整个系统最基础的配置。
\par 第二章详细介绍了在树莓派平台上安装GNU Radio的方法,着重介绍了GNU Radio程序编写过程中使用的一些库和软件,如果后续开发者需要进行源码的修改,这些知识必不可少。
\par 第三章简要的说明了DVB-T系统的结构与原理,通过框图解释了整个系统的运行过程,方便读者对整个系统有个总体的了解。
\par 第四章重点介绍了DVB-T发射端的原理,相关模块的参数,同时给出了相关的源码,方便读者理解代码逻辑。
\par 第五章给出了DVB-T在GNU Radio上的实现方案,及设备连接示意图。
\par 第六章通过星座图、频谱等展示了整个系统的性能,也指出了整个系统的不足、可以改进之处。
\par 第七章总结了整个文章,做出了评价与展望。

\clearpage