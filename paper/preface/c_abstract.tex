%!TEX root = ../paper.tex
\renewcommand{\baselinestretch}{1.5}
\fontsize{12pt}{13pt}\selectfont

\chapter*{摘~~~~要}
\markboth{中~文~摘~要}{中~文~摘~要}

\par 软件无线电与DVB-T技术都已经发展了许多年,将二者整合起来并搭载到一个相对便捷的平台上以实现诸如测试、自定义编码格式、发射指定信号等特定用途,不失为一项具有广泛应用前景的课题。
\par 本论文将通过GNU Radio平台搭建一个可以稳定传输的DVB-T系统,研究这套系统在相对便捷的ARM平台(树莓派3B+)上实现的可行性。
\par 本论文将从树莓派的使用开始,介绍GNU Radio使用的一些库和软件,其中主要介绍了CMake和git的使用,以减少后续开发者研究、开发的难度。之后简要介绍了DVB-T系统的各个模块的参数,并在相关模块中给出了相应的代码实现。
\par 最后给出了一种在GNU Radio平台上搭建DVB-T系统的实现方法,通过频谱说明整个系统的性能,指出了不足及可以改进之处。

\vspace{1em}
\noindent {\fHei 关键词:} \quad GNU Radio,DVB-T,OFDM,树莓派

\clearpage
\endinput
