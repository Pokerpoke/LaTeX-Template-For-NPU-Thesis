%!TEX root = ../paper.tex
\renewcommand{\baselinestretch}{1.5}
\fontsize{12pt}{13pt}\selectfont

\chapter*{Abstract}
\markboth{英~文~摘~要}{英~文~摘~要}

\par\noindent Software radio and DVB-T technology have been developed for many years, the two together and carried on a relatively convenient platform to achieve such as measurement, custom coding format, launch the specified signal and other specific purposes, because GNU Radio.
\par\noindent This paper will build a DVB-T system that can be transmitted stably on the GNU Radio platform to study the feasibility of this system on a relatively convenient ARM platform--Raspberry Pi 3B+. 
\par\noindent This paper will start with the use of Raspberry Pi, GNU Radio will introduce some of the library and software, mainly describes the use of CMake and git to reduce the difficulty of follow-up developers research and development. After that, the parameters of each module of DVB-T system are briefly introduced, and the code is given in the relevant module. 
\par\noindent Finally, an implementation method of constructing DVB-T system on GNU Radio platform is given, and the performance of the whole system is explained by spectrum, and the deficiency and improvement can be pointed out.Software radio and DVB-T technology have been developed for many years, the two together and carried on a relatively convenient platform to achieve such as measurement, custom coding format, launch the specified signal and other specific purposes, because GNU Radio This paper will build a DVB-T system that can be transmitted stably through the GNU Radio platform to study the feasibility of this system on a relatively convenient ARM platform (Raspberry 3B +). This paper will start with the use of raspberry, GNU Radio will introduce some of the library and software, mainly describes the use of CMake and git to reduce the difficulty of follow-up developers research and development. After that, the parameters of each module of DVB-T system are briefly introduced, and the code is given in the relevant module. Finally, an implementation method of constructing DVB-T system on GNU Radio platform is given, and the performance of the whole system is explained by spectrum, and the deficiency and improvement can be pointed out.

\noindent To sum up, this paper works on those

\vspace{1em}
\noindent {\textbf{Key Words:}} \quad GNU Rasio, DVB-T, OFDM, Raspberry Pi

\clearpage
\endinput