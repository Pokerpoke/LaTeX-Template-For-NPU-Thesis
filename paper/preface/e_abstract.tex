%!TEX root = ../paper.tex
\renewcommand{\baselinestretch}{1.5}
\fontsize{12pt}{13pt}\selectfont

\chapter*{Abstract}
\markboth{英~文~摘~要}{英~文~摘~要}

\par Software radio and DVB-T technology have been developed for many years, the two together and carried on a relatively convenient platform to achieve such as testing, custom information format, transmit the specified signal and other specific purposes.This will be a widely used prospect.
\par This paper will build a DVB-T system that can transmit stably on the GNU Radio platform to study the feasibility of this system on a relatively convenient ARM platform--Raspberry Pi 3B+.
\par This paper start with the use of Raspberry Pi.Followed by the introduction of GNU Radio, whick introduces some libraries and software, mainly describes the use of CMake and git to reduce the difficulty for follow-up developers for further research and development. After that, the parameters of each module of DVB-T system are briefly introduced. And the main code of some modules are given. 
\par Finally, an design of constructing DVB-T system on GNU Radio platform is given.The performance of the whole system is explained by spectrum, and the deficiency and improvement can be pointed out.

% \noindent To sum up, this paper works on those

\vspace{1em}
\noindent {\textbf{Key Words:}} \quad GNU Radio, DVB-T, OFDM, Raspberry Pi

\clearpage
\endinput