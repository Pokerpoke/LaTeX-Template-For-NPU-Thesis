%!TEX root = ../example.tex
\chapter{Linux安装}
	\section{x86平台}
		\subsection{虚拟机}
			\par VMware
			\par Hyper-V
		\subsection{双系统}
			\par 空间分配
				\begin{itemize}
					\item /
					\item /swap
					\item /boot
					\item /home
				\end{itemize}
			\par LVM
			\par 快照
		\subsection{安装}
		\subsection{软件源}
			\begin{itemize}
				\item 新立得包管理器
				\item 阿里云
			\end{itemize}
	\section{ARM平台---树莓派}
		\subsection{简介}
		\par 树莓派(英语:Raspberry Pi),是一款基于Linux的单板机电脑。它由英国的树莓派基金会所开发,目的是以低价硬件及自由软件促进学校的基本计算机科学教育。
		\par 树莓派的生产是通过有生产许可的两家公司:Element 14/Premier Farnell和RS Components。这两家公司都在网上出售树莓派。
		\par 树莓派配备一枚博通(Broadcom)出产的ARM架构700MHz BCM2835处理器,256MB內存(B型已升级到512MB内存),使用SD卡当作存储媒体,且拥有一个Ethernet、两个USB接口、以及HDMI(支持声音输出)和RCA端子输出支持。树莓派只有一张信用卡大小,体积大概是一个火柴盒大小,可以运行像《雷神之锤III竞技场》的游戏和进行1080p视频的播放。操作系统采用开源的Linux系统如Debian、ArchLinux,自带的Iceweasel、KOffice等软件,能够满足基本的网络浏览、文字处理以及电脑学习的需要。分A、B两种型号,售价分别是A型25美元、B型35美元。树莓派基金会从2012年2月29日开始接受B型的订货。
		\par 树莓派基金会提供了基于ARM架构的Debian、Arch Linux和Fedora等的发行版供大众下载,还计划提供支持Python作为主要编程语言,支持BBC BASIC、C语言和Perl等编程语言。
		树莓派基金会于2016年2月发布了树莓派3,较前一代树莓派2,树莓派3的处理器升级为了64位的博通BCM2837,并首次加入了Wi-Fi无线网络及蓝牙功能,而售价仍然是35美元。\cite{ wiki:树莓派}
		\par\noindent $\bullet$ 树莓派3B+参数
		\begin{table}[!hbp]		%开始表格
								%其中参数[!hbp] 的意思是:
								%!表示尽可能的尝试 h(here) 当前位置显示表格,
								%如果实在不行显示在 b(bottom) 底部,
		\begin{tabular}{l|p{0.8\columnwidth}}
		\hline
		\hline
		SoC & Broadcom\ BCM2837(CPU,GPU\ DSP和SDRAM、USB)\\
		\hline
		CPU & ARM\ Cortex-A53\ 64位\ (ARMv8系列)\ 1.2GHz\ (四核心)\\
		\hline
		GPU & Broadcom\ VideoCore\ IV[43],\ OpenGL\ ES\ 2.0,1080p\ 30\ h.264/MPEG-4\ AVC高清解码器\\
		\hline
		RAM & 1024\ MB\ (LPDDR2)\\
		\hline
		外设 & 14个GPIO及HAT规格铺设\\
		\hline
		额定功率 & 4.0\ 瓦\ (5V/800mA)\\
		\hline
		电源输入 & 5V\ 电压\ (通过MicroUSB或经GPIO输入)\\
		\hline
		总体尺寸 & 85.60\ ×\ 53.98\ 毫米(3.370\ ×\ 2.125\ 英寸)\\
		\hline
		重量 & 45\ g(1.6\ oz)\\
		\hline
		\end{tabular}
		\end{table}
		\subsection{烧录系统镜像}
		\subsection{软件源}
		\subsection{无线网络配置}
		\subsection{镜像备份}
\begin{figure}[ht]
	\centering
	\includegraphics[scale=0.6]{figures/figure1.png}
	\caption{
		这里是个普通的标题
	}
	\label{fig:example}
\end{figure}
% \endinput