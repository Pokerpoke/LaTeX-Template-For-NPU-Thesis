\chapter{环境配置}
\section{CMake}
	\subsection{简介}
	\par CMake是个开源的跨平台自动化建构系统,它用配置文件控制建构过程(build process)的方式和Unix的Make相似,只是CMake的配置文件取名为CMakeLists.txt。Cmake并不直接建构出最终的软件,而是产生标准的建构档(如Unix的Makefile或Windows Visual C++的projects/workspaces),然后再依一般的建构方式使用。这使得熟悉某个集成开发环境(IDE)的开发者可以用标准的方式建构他的软件,这种可以使用各平台的原生建构系统的能力是CMake和SCons等其他类似系统的区别之处。CMake可以编译源代码、制做程序库、产生适配器(wrapper)、还可以用任意的顺序建构可执行文件。CMake支持in-place建构(二进档和源代码在同一个目录树中)和out-of-place建构(二进档在别的目录里),因此可以很容易从同一个源代码目录树中建构出多个二进档。CMake也支持静态与动态程序库的建构。\cite{ wiki:CMake}
	\subsection{安装}
	\par CMake要求每个目录下均有一个CMakeLists.txt文件,一个通用的目录结构如下:
	\dirtree{%
	.1 ..
	.2 app.
	.3 CMakeLists.txt.
	.3 main.cc.
	.3 main.h.
	.2 include.
	.3 CMakeLists.txt.
	.3 test.h.
	.2 lib.
	.3 test\_impl.cc.
	.3 test\_impl.h.
	.2 CMakeLists.txt.
	.2 README.md.	
	}
\section{Boost C++ Libraries}
	\subsection{简介}
	\par Boost C++ 库(Libraries)是一组扩充C++功能的经过同行评审(Peer-reviewed)且开放源代码程序库。大多数的函数为了能够以开放源代码、封闭项目的方式运作,而授权于Boost软件许可协议(Boost Software License)之下。许多Boost的开发人员是来自C++标准委员会,而部分的Boost库成为C++的TR1标准之一。
	\par 为了要确保库的效率与弹性,Boost广泛的使用模板(template)功能。而它是针对各式领域的C++用户与应用领域(Application Domain)上,包含的库类别从像smart\_ptr 库这种类通用库,到像是文件系统的操作系统抽象层,甚至能够利用Boost来开发额外的库或是给高级的C++用户利用,像是MPL。\cite{ wiki:Boost}
\section{Doxygen}
	\subsection{简介}
	\par Doxygen是一个C++、C、Java、Objective-C、Python、IDL(CORBA和Microsoft flavors)、Fortran、VHDL、PHP、C\#和D语言的文檔生成器。可以在大多数类Unix的系统上运行,以及Mac OS X操作系统和Microsoft Windows。初始版本的Doxygen使用了一些旧版本DOC++的源代码;随后,Doxygen源代码由Dimitri van Heesch重写。
	\par Doxygen是一个编写软件参考文檔的工具。该文檔是直接写在源代码中,因此比较容易保持更新。Doxygen可以交叉引用文檔和源代码,使文件的读者可以很容易地引用实际的源代码。
	\par KDE使用Doxygen作为其部分文档且KDevelop具有内置的支持。 Doxygen的发布遵守GNU通用公共许可证,并且是自由软件。\cite{ wiki:Doxygen}
\section{Python}
	\subsection{简介}
	\par Python(英国发音:/ˈpaɪθən/ 美国发音:/ˈpaɪθɑːn/),是一种面向对象、解释型的计算机程序语言。它包含了一组功能完备的标准库,能够轻松完成很多常见的任务。它的语法简单,与其它大多数程序设计语言使用大括号不一样,它使用缩进来定义语句块。
	\par 与Scheme、Ruby、Perl、Tcl等动态语言一样,Python具备垃圾回收功能,能够自动管理内存使用。它经常被当作脚本语言用于处理系统管理任务和网络程序编写,然而它也非常适合完成各种高级任务。Python虚拟机本身几乎可以在所有的作业系统中运行。使用一些诸如py2exe、PyPy、PyInstaller之类的工具可以将Python源代码转换成可以脱离Python解释器运行的程序。
	\par Python的官方解释器是CPython,该解释器用C语言编写,是一个由社区驱动的自由软件,目前由Python软件基金会管理。
	\par Python支持命令式程序设计、面向对象程序设计、函数式编程、面向侧面的程序设计、泛型编程多种编程范式。\cite{ wiki:Python}
\section{SWIG}
	\subsection{简介}
	\par SWIG是个帮助使用C或者C++编写的软件能与其它各种高级编程语言进行嵌入联接的开发工具。SWIG能应用于各种不同类型的语言包括常用脚本编译语言例如Perl, PHP, Python, Tcl, Ruby and PHP。支持语言列表中也包括非脚本编译语言,例如C\#, Common Lisp (CLISP, Allegro CL, CFFI, UFFI), Java, Modula-3, OCAML以及R,甚至是编译器或者汇编的计划应用(Guile, MzScheme, Chicken)。SWIG普遍应用于创建高级语言解析或汇编程序环境,用户接口,作为一种用来测试C/C++或进行原型设计的工具。SWIG还能够导出XML或Lisp s-expressions格式的解析树。SWIG可以被自由使用,发布,修改用于商业或非商业中。\cite{ wiki:SWIG}
\section{USRP}
	\subsection{简介}
	\par Universal Software Radio Peripheral (USRP) is a range of software-defined radios designed and sold by Ettus Research and its parent company, National Instruments. Developed by a team led by Matt Ettus, the USRP product family is intended to be a comparatively inexpensive hardware platform for software radio, and is commonly used by research labs, universities, and hobbyists.
	\par Most USRPs connect to a host computer through a high-speed link, which the host-based software uses to control the USRP hardware and transmit/receive data. Some USRP models also integrate the general functionality of a host computer with an embedded processor that allows the USRP device to operate in a stand-alone fashion.
	\par The USRP family was designed for accessibility, and many of the products are open source hardware. The board schematics for select USRP models are freely available for download; all USRP products are controlled with the open source UHD driver, which is free and open source software. USRPs are commonly used with the GNU Radio software suite to create complex software-defined radio systems. \cite{ wiki:USRP}
\endinput