%!TEX root = ../paper.tex
\chapter{DVB-T发射端}
	\section{能量扩散}
		\par 当发射的信号过于集中在某一频段上,将对与其共用频段的其他系统或设备造成较大干扰。人为地对发射信号进行随机化或者加扰,可以使原本集中的信号能量均匀分布。同时,由于添加的随机码为伪随机码,所以在接收端很容易进行去随机化。在数字通信过程中,通过添加伪随机信号还能缩短连续的“0”或者连续的“1”的长度,接收端提取定时信号将变得更加容易。
		\par MPEG-2的传输复用包长为188字节,包括一个同步字节,如图\ref{fig:MPEG_2_TS}所示。\cite{数字电视DVB标准能量扩散的FPGA设计与实现_肖闽进}
		\begin{figure}[thb]
	\centering
	\begin{tikzpicture}
		\node(SYNC)[rectangle,draw=black,text centered,minimum height=1cm,minimum width=2cm,label=below:1字节]{SYNC};
		\node(data)[rectangle,draw=black,text centered,right = 0 of SYNC,minimum height=1cm,minimum width=8cm,label=below:187字节]{MPEG-2传输数据};
	\end{tikzpicture}
	\caption{MPEG-2传输包}
	\label{fig:MPEG_2_TS}
\end{figure}
\endinput
		\par 其中同步字节为0x47,传送时从高位开始送入即(0100 0111)的0开始送入,由伪随机二进制(PRBS,Pseudo Random Binary Sequence)和码流数据按位异或完成,结构如图\ref{fig:PRBS}所示。
		\begin{figure}[thb]
	\centering
	\begin{tikzpicture}[circuit logic US]
		\node (data)[rectangle split,rectangle split horizontal,rectangle split parts=15,draw,text width=0.4cm] at (0,0){
			\nodepart{text}1\nodepart{two}2\nodepart{three}3\nodepart{four}4\nodepart{five}5\nodepart{six}6\nodepart{seven}7\nodepart{eight}8\nodepart{nine}9\nodepart{ten}10\nodepart{eleven}11\nodepart{twelve}12\nodepart{thirteen}13\nodepart{fourteen}14\nodepart{fifteen}15
		};
		\node (comment1)[above=0.1cm of data,rectangle split,rectangle split horizontal,rectangle split parts=15,text width=0.4cm]{
			\nodepart{text}1\nodepart{two}0\nodepart{three}0\nodepart{four}1\nodepart{five}0\nodepart{six}1\nodepart{seven}0\nodepart{eight}1\nodepart{nine}0\nodepart{ten}0\nodepart{eleven}0\nodepart{twelve}0\nodepart{thirteen}0\nodepart{fourteen}0\nodepart{fifteen}0
		};
		\node (xor1)[xor gate,point left] at (3,-1){};
		\node (xor2)[xor gate,point right] at (2,-2.1){};
		\node (and)[and gate,point right] at (0,-2){};
		\node (enable) at (-1,-3) {使能};
		\node (input) at (1,-3.5) {输入数据};
		\node (output) at (5,-2.1) {随机化后的数据};
		\node (comment2) at (1,-1.3) {0 0 0 0 0 0 1 1......};
		
		\draw (data.fourteen south) |- (xor1.input 2);
		\draw (data.east) -| ++(1,-1) |- (xor1.input 1);
		\draw (data.text west) -| (-0.8*8,-1) |- (xor1.output);
		\draw (-0.8*8,-1) |- (and.input 1);
		\draw (enable.north) |- (and.input 2);
		\draw (and.output) -- (xor2.input 1);
		\draw (input.north) |- (xor2.input 2);
		\draw (xor2.output) -- (output.west);
		\fill (-0.8*8,-1) circle (2pt);
	\end{tikzpicture}
	\caption{能量扩散原理图}
	\label{fig:PRBS}
\end{figure}
\endinput
		\par PRBS的生成多项式为$g(x)=1+X^{14}+X^{15}$,伪随机寄存器中的初始序列是“1001 0101 0000 000”,每8个传输包初始化一次。同时,为了方便接收,每8个传输包为一组,并对每8个传输包的第一个包的同步字节按比特取反,即由0x47变为0xB8。随机化从第1个传输包的同步字节sync后的第1个比特开始进行,但在随后的7个传输包中的同步字节,继续产生伪随机码,但是不对输入的同步字节进行加扰,同步字节保持原状,这样PRBS的周期为$188*8-1=1503$字节。输入码流中断或者不是MPEG-2 TS码流格式时,随机化继续进行,插入空白字节与同步字节完成空包处理,接收端识别全零的空包并将其删除。
		\par 主要程序实现如代码段\ref{code:prbs}:
		\begin{lstlisting}[caption = {能量扩散}, label = {code:prbs}, language = C++ ]
void dvbt_energy_dispersal_impl::clock_prbs(int clocks)
{
	int res = 0;
	int feedback = 0;

	for (int i = 0; i < clocks; i++) {
	feedback = ((d_reg >> (14 - 1)) ^ (d_reg >> (15 - 1))) & 0x1;
	d_reg = ((d_reg << 1) | feedback) & 0x7fff;
	res = (res << 1) | feedback;
	}
	return res;
}
		\end{lstlisting}
	\section{RS编码}
		\par DVB-T的外编码采用RS编码(Reed-Solomon)编码,RS编码是一种线性分组循环码,以长度为$n$的一组符号为单位处理,组中的$n$个符号由$k$个欲传输的信息符号按一定关系生成的,RS编码具有极强的随机错误和突发错误纠正能力。
		\par DVB-T系统使用的是由RS(255,239,$t$=8)衍生出的删余RS(204,188,$t$=8)码,通过对每个随机化后的188字节的传输包编码,通过生成多项式,生成一个误码保护包,RS编码从同步字节(0x47)或者倒相后的同步字节(0xB8)开始,误码保护包如图\ref{fig:rs_204_188}。\cite{RS编码器的设计与实现_游余新}
		\begin{figure}[thb]
	\centering
	\begin{tikzpicture}
		\node (sync)[rectangle,draw] at (0,0) {$\overline{\text{SYNC}}or\text{SYNC}_n$(1字节)};
		\node (data)[rectangle,draw,right=0 of sync]{随机化数据(187字节)};
		\node (check)[rectangle,draw,right=0 of data]{奇偶检验位(16字节)};
		\node (comment1) at (5,-1){204字节};
		\node (arrow1)[below = 0.1cm of sync.south west]{};
		\node (arrow2)[below = 0.1cm of check.south east]{};

		\draw [|<->|](arrow1) -- (arrow2);
	\end{tikzpicture}
	\caption{RS(204,188,8)误码保护包}
	\label{fig:rs_204_188}
\end{figure}
\endinput
		\par 可以看到,编码后的总长度为204字节,有效数据为188字节,最多可以校正8个字节的随机误码错误。
		\par RS码生成多项式为:
		\begin{equation}
			g(x)=(x+\lambda ^0)(x+\lambda ^1)(x+\lambda ^2)...(x+\lambda ^{15})
		\end{equation}
		\par 其中$\lambda$=0x02。
		\par 域生成多项式为:
		\begin{equation}
			p(x)=x^8+x^4+x^3+x^2+1
		\end{equation}
		\par 如果想要通过RS(255,239,$t$=8)编码器,可以在传输包前面增加51个全零字节,经过编码程序以后,在将这些无用的全零字节删除,最终产生$N$=204字节的RS码字。
		\par RS编码初始化见代码段\ref{code:rs_init},RS编码见代码段\ref{code:rs_encode}。
		\begin{lstlisting}[caption = {RS编码初始化}, label={code:rs_init}, language = C++ ]
void dvbt_reed_solomon::rs_init(int lambda, int n, int k, int t)
{
	d_n = n; d_k = k; d_t = t;
	// 2t = n - k, dmin = 2t + 1 = n -k + 1

	d_l = new unsigned char[d_n + 1];
	if (d_l == NULL) {
	std::cout << "Cannot allocate memory for d_l" << std::endl;
	exit(1);
	}

	d_g = new unsigned char[2 * d_t + 1];
	if (d_g == NULL) {
	std::cout << "Cannot allocate memory for d_g" << std::endl;
	delete [] d_l;
	exit(1);
	}

	//Generate roots of lambda
	d_l[0] = 1;

	for (int i = 1; i <= d_n; i++) {
	d_l[i] = gf_mul(d_l[i - 1], lambda);
	}

	//Init Generator polynomial buffer
	for (int i = 0; i <= (2*t); i++) {
	d_g[i] = 0;
	}

	//Start with x+lambda^0
	d_g[0] = 1;

	//Create generator polynomial
	for (int i = 1; i <= (2 * t); i++) {
	for (int j = i; j > 0; j--) {
		if (d_g[j] != 0) {
		d_g[j] = gf_add(d_g[j - 1], gf_mul(d_g[j], d_l[i - 1]));
		}
		else {
		d_g[j] = d_g[j - 1];
		}
	}

	d_g[0] = gf_mul(d_g[0], d_l[i - 1]);
	}

	// Init syndrome array
	d_syn = new unsigned char[2 * d_t + 1];
	if (d_syn == NULL) {
	std::cout << "Cannot allocate memory for d_syn" << std::endl;
	delete [] d_g;
	delete [] d_l;
	exit(1);
	}
}
		\end{lstlisting}
		\begin{lstlisting}[caption = {RS编码}, label = {code:rs_encode}, language = C++ ]
int dvbt_reed_solomon::rs_encode(unsigned char *data_in, unsigned char *parity)
{
	memset(parity, 0, 2 * d_t);

	for (int i = 0; i < d_k; i++) {
	int feedback = gf_add(data_in[i], parity[0]);

	if (feedback != 0) {
		for (int j = 1; j < (2 * d_t); j++) {
		if (d_g[2 * d_t - j] != 0) {
			parity[j] = gf_add(parity[j], gf_mul(feedback, d_g[2 * d_t - j]));
		}
		}
	}

	//Shift the register
	memmove(&parity[0], &parity[1], (2 * d_t) - 1);

	if (feedback != 0) {
		parity[2 * d_t - 1] = gf_mul(feedback, d_g[0]);
	}
	else {
		parity[2 * d_t - 1] = 0;
	}
	}

	return (0);
}
		\end{lstlisting}
	\section{卷积交织}
		\par DVB-T系统的外交织采用的是深度$I$=12,基数为17的,基于字节的卷积交织,现用一个$M$=3的例子说明,此时卷积交织工作过程如图\ref{fig:convolutional_interleaver}所示。第一行直通,第二行有$M$=3个存储器,第三行有$2M$=6个存储器,输入、输出的每个字节由同步开关进行控制,第二行延时$M$个字节,第三行延时$2M$个字节,解交织的过程与此相反。输入输出数据流如图\ref{fig:convolutional_interleaver_data_structure}所示\cite{用FPGA实现DVB标准中的卷积交织_刘静},这样就将原来的数据分散发送了。
		\begin{figure}[thb]
	\centering
	\begin{tikzpicture}[align=center]
		\node (inputM3)[draw,text width=1.2cm] at (2,0) {M=3};
		\node (inputM6)[draw,text width=1.2cm] at (2,-1) {2M=6};

		\draw [arrow] (-1,0) -- (0,0) -- (1,1);
		\draw [arrow] (1,1) -- (3,1);
		\draw [arrow] (3,1) -- (4,0) -- (5,0);
		\draw [arrow] (0.75,0) -- (inputM3.west);
		\draw [arrow] (inputM3.east) -- (3.25,0);
		\draw [arrow] (0.75,-1) -- (inputM6.west);
		\draw [arrow] (inputM6.east) -- (3.25,-1);
	\end{tikzpicture}
	\caption{M=3时卷积交织图}
	\label{fig:convolutional_interleaver}
\end{figure}
\endinput
		\begin{figure}[thb]
	\centering
	\begin{tikzpicture}[align=center]
		\node (input1)[] at (0,2) {0,3,6};
		\node (input2)[] at (0,1) {1,4,7};
		\node (input3)[] at (0,0) {2,5,8};
		\node (output1)[] at (5,2) {0};
		\node (output2)[] at (5,1) {x};
		\node (output3)[] at (5,0) {x};
		\node (M3)[draw,text width=1.5cm] at (2.5,1) {M=3};
		\node (M6)[draw,text width=1.5cm] at (2.5,0) {2M=6};
		
		\node at (5.5,2) {3};
		\node at (6,2) {6};
		\foreach \x in {6.5,7,...,9.5}{
			\node at (\x,2) {\&};
		};
		\foreach \x in {5.5,6}{
			\node at (\x,1) {x};
		};
		
		\node at (6.5,1) {1};
		\node at (7,1) {4};
		\node at (7.5,1) {7};
		\foreach \x in {8,8.5,...,9.5}{
			\node at (\x,1) {\&};
		};
		
		\foreach \x in {5.5,6,...,7.5}{
			\node at (\x,0) {x};
		};
		\node at (8,0) {2};
		\node at (8.5,0) {5};
		\node at (9,0) {8};
		\node at (9.5,0) {\&};
		
		\draw [arrow](input1) -- (output1);
		\draw [arrow](input2) -- (M3.west);
		\draw [arrow](M3.east) -- (output2);
		\draw [arrow](input3) -- (M6.west);
		\draw [arrow](M6.east) -- (output3);
		
		\foreach \x in {5.25,5.75,...,9.25}{
			\draw [arrow](\x,2.5) -- (\x,-0.5);
		};
		
		\node at (0,-1) {输入数据流};
		\node at (6,-1) {输出数据流};

	\end{tikzpicture}
	\caption{卷积交织输入输出数据结构}
	\label{fig:convolutional_interleaver_data_structure}
\end{figure}
\endinput
		\par DVB-T中使用的卷积交织,$M$=17,即第一行直通,第二行延时17字节,第三行延时2*17=34字节,以此类推,总过有12层,同步字节($\overline{\text{SYNC}}$和$\text{SYNC}_n$)总是通过第一行进行交织。
		\par 主要程序实现见代码段\ref{code:convolutional_interleaver}。
		\begin{lstlisting}[caption = {卷积交织},label = {code:convolutional_interleaver},language = C++ ]
int dvbt_convolutional_interleaver_impl::work (int noutput_items,
				gr_vector_const_void_star &input_items,
				gr_vector_void_star &output_items)
{
	const unsigned char *in = (const unsigned char *) input_items[0];
	unsigned char *out = (unsigned char *) output_items[0];

	for (int i = 0; i < (noutput_items / d_I); i++) {
	//Process one block of I symbols
	for (unsigned int j = 0; j < d_shift.size(); j++) {
		d_shift[j]->push_front(in[(d_I * i) + j]);
		out[(d_I * i) + j] = d_shift[j]->back();
		d_shift[j]->pop_back();
	}
	}
	return noutput_items;
}
		\end{lstlisting}
	\section{内编码(卷积编码)}
		\par DVB-T系统的内编码采用卷积编码,卷积码是由$k$个信息比特编码成$n(n>k)$比特码组,编码出的l比特的码组值,不仅与当前码字中的k个信息比特值有关,而且与其前面$N-1$个码字中的$(N-1)xk$个信息比特值有关,也即当前码组内的$n$个码元,它们的值取决于$N$个码组内的全部信息码元,$N$可称为卷积码编码的约束长度。
		\par DVB-T系统可以选择几种由码率为1/2的主卷积码删余后的卷积码。无论是等级或非等级传送模式,对于某一给定业务的数据率,系统可选择最适当的码率。主码的生成多项式,对$X$路输出是$G_1=171_{OCT}$,对$Y$路输出是$G_2=133_{OCT}$。图\ref{fig:inner_coder}是主卷积码生成结构图。
		\begin{figure}[thb]
	\centering
	\begin{tikzpicture}[align=center]
		\node (and1) [] at (0,3) {$\bigoplus$};
		\node (and2) [] at (0,-3) {$\bigoplus$};
		\node (pro1) [draw,text width=3em] at (-5,0) {1比特延迟};
		\node (pro2) [draw,text width=3em] at (-3,0) {1比特延迟};
		\node (pro3) [draw,text width=3em] at (-1,0) {1比特延迟};
		\node (pro4) [draw,text width=3em] at (1,0) {1比特延迟};
		\node (pro5) [draw,text width=3em] at (3,0) {1比特延迟};
		\node (pro6) [draw,text width=3em] at (5,0) {1比特延迟};

		\draw [arrow] (-6.5,0) -- (pro1.west);
		\draw [arrow] (pro1.east) -- (pro2.west);
		\draw [arrow] (pro2.east) -- (pro3.west);
		\draw [arrow] (pro3.east) -- (pro4.west);
		\draw [arrow] (pro4.east) -- (pro5.west);
		\draw [arrow] (pro5.east) -- (pro6.west);
		\draw (pro6.east) -- (6,0);

		\draw [arrow] (-6,0) -- (-6,1) -- (-0.4,2.8);
		\draw [arrow] (-4,0) -- (-4,1) -- (-0.2,2.8);	
		\draw [arrow] (-2,0) -- (-2,1) -- (-0.1,2.8);
		\draw [arrow] (0,0) -- (0,2.8);
		\draw [arrow] (6,0) -- (6,1) -- (0.4,2.8);

		\draw [arrow] (-6,0) -- (-6,-1) -- (-0.4,-2.8);
		\draw [arrow] (-2,0) -- (-2,-1) -- (-0.2,-2.8);
		\draw [arrow] (0,0) -- (-0,-2.8);
		\draw [arrow] (4,0) -- (4,-1) -- (0.2,-2.8);
		\draw [arrow] (6,0) -- (6,-1) -- (0.1,-2.8);

		\fill (-6,0) circle (2pt);
		\fill (-4,0) circle (2pt);
		\fill (-2,0) circle (2pt);
		\fill (2,0) circle (2pt);
		\fill (4,0) circle (2pt);
		\fill (6,0) circle (2pt);


	\end{tikzpicture}
	\caption{1/2码率主卷积码}
	\label{fig:inner_coder}
\end{figure}
\endinput
	\section{比特交织}
	\section{符号交织}
	\section{星座映射}
	\section{参考信号}
	\section{IFFT}
	\section{OFDM循环前缀}
	\section{常数}
	\section{重采样}
		\par 信号的采样率,用于满足另一个系统的要求
	\section{USRP发射}
