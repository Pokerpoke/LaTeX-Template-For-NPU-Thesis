%!TEX root = ../paper.tex
\chapter{DVB-T发射端}
	\section{能量扩散}
		\par 当发射的信号过于集中在某一频段上,将对与其共用频段的其他系统或设备造成较大干扰。人为地对发射信号进行随机化或者加扰,可以使原本集中的信号能量均匀分布。同时,由于添加的随机码为伪随机码,所以在接收端很容易进行去随机化。在数字通信过程中,通过添加伪随机信号还能缩短连续的“0”或者连续的“1”的长度,接收端提取定时信号将变得更加容易。
		\par MPEG-2的传输复用包长为188字节,包括一个同步字节,如图\ref{fig:MPEG_2_TS}所示。\cite{数字电视DVB标准能量扩散的FPGA设计与实现}
		\begin{figure}[thb]
	\centering
	\begin{tikzpicture}
		\node(SYNC)[rectangle,draw=black,text centered,minimum height=1cm,minimum width=2cm,label=below:1字节]{SYNC};
		\node(data)[rectangle,draw=black,text centered,right = 0 of SYNC,minimum height=1cm,minimum width=8cm,label=below:187字节]{MPEG-2传输数据};
	\end{tikzpicture}
	\caption{MPEG-2传输包}
	\label{fig:MPEG_2_TS}
\end{figure}
\endinput
		\par 其中同步字节为0x47,传送时从高位开始送入即(0100 0111)的0开始送入,由伪随机二进制(PRBS,Pseudo Random Binary Sequence)和码流数据按位异或完成,结构如图\ref{fig:PRBS}所示。
		\begin{figure}[thb]
	\centering
	\begin{tikzpicture}[circuit logic US]
		\node (data)[rectangle split,rectangle split horizontal,rectangle split parts=15,draw,text width=0.4cm] at (0,0){
			\nodepart{text}1\nodepart{two}2\nodepart{three}3\nodepart{four}4\nodepart{five}5\nodepart{six}6\nodepart{seven}7\nodepart{eight}8\nodepart{nine}9\nodepart{ten}10\nodepart{eleven}11\nodepart{twelve}12\nodepart{thirteen}13\nodepart{fourteen}14\nodepart{fifteen}15
		};
		\node (comment1)[above=0.1cm of data,rectangle split,rectangle split horizontal,rectangle split parts=15,text width=0.4cm]{
			\nodepart{text}1\nodepart{two}0\nodepart{three}0\nodepart{four}1\nodepart{five}0\nodepart{six}1\nodepart{seven}0\nodepart{eight}1\nodepart{nine}0\nodepart{ten}0\nodepart{eleven}0\nodepart{twelve}0\nodepart{thirteen}0\nodepart{fourteen}0\nodepart{fifteen}0
		};
		\node (xor1)[xor gate,point left] at (3,-1){};
		\node (xor2)[xor gate,point right] at (2,-2.1){};
		\node (and)[and gate,point right] at (0,-2){};
		\node (enable) at (-1,-3) {使能};
		\node (input) at (1,-3.5) {输入数据};
		\node (output) at (5,-2.1) {随机化后的数据};
		\node (comment2) at (1,-1.3) {0 0 0 0 0 0 1 1......};
		
		\draw (data.fourteen south) |- (xor1.input 2);
		\draw (data.east) -| ++(1,-1) |- (xor1.input 1);
		\draw (data.text west) -| (-0.8*8,-1) |- (xor1.output);
		\draw (-0.8*8,-1) |- (and.input 1);
		\draw (enable.north) |- (and.input 2);
		\draw (and.output) -- (xor2.input 1);
		\draw (input.north) |- (xor2.input 2);
		\draw (xor2.output) -- (output.west);
		\fill (-0.8*8,-1) circle (2pt);
	\end{tikzpicture}
	\caption{能量扩散原理图}
	\label{fig:PRBS}
\end{figure}
\endinput
		\par PRBS的生成多项式为$g(x)=1+X^{14}+X^{15}$,伪随机寄存器中的初始序列是“1001 0101 0000 00000”,每8个传输包初始化一次。同时,为了方便接收,每8个传输包为一组,并对每8个传输包的第一个包的同步字节按比特取反,即由0x47变为0xB8。随机化从第1个传输包的同步字节sync后的第1个比特开始进行,但在随后的7个传输包中的同步字节,继续产生伪随机码,但是不对输入的同步字节进行加扰,同步字节保持原状,这样PRBS的周期为$188*8-1=1503$字节。
		
	\section{RS编码}
	\section{卷积交织}
	\section{内编码(卷积编码)}
	\section{比特交织}
	\section{符号交织}
	\section{星座映射}
	\section{参考信号}
	\section{IFFT}
	\section{OFDM循环前缀}
	\section{常数}
	\section{重采样}
		\par 信号的采样率,用于满足另一个系统的要求
	\section{USRP发射}
