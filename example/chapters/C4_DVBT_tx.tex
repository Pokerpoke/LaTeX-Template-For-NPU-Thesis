%!TEX root = ../paper.tex
\chapter{DVB-T发射端}

	\section{能量扩散}
		\par 当发射的信号过于集中在某一频段上,将对与其共用频段的其他系统或设备造成较大干扰。人为地对发射信号进行随机化或者加扰,可以使原本集中的信号能量均匀分布。同时,由于添加的随机码为伪随机码,所以在接收端很容易进行去随机化。在数字通信过程中,通过添加伪随机信号还能缩短连续的“0”或者连续的“1”的长度,接收端提取定时信号将变得更加容易。
	\section{RS编码}
	\section{卷积交织}
	\section{内编码(卷积编码)}
	\section{比特交织}
	\section{符号交织}
	\section{星座映射}
	\section{参考信号}
	\section{IFFT}
	\section{OFDM循环前缀}
	\section{常数}
	\section{重采样}
		\par 信号的采样率,用于满足另一个系统的要求
	\section{USRP发射}
