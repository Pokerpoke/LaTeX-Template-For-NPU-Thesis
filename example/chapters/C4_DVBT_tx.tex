%!TEX root = ../paper.tex
\chapter{DVB-T发射端}
	\section{DVB-T系统结构与原理}
		\par 地面数字视频广播(英语:Digital Video Broadcasting-Terrestrial,缩写:DVB-T),是欧洲广播联盟在1997年发布的数字地面电视视频广播传输,最早是1998年在英国实行广播。
		\par 2007年,欧洲DVB组织推出改良版的DVB-T2地面数字电视广播标准,频谱利用率及有效传输码率得到较大提高。 DVB-T2先从没有采用DVB-T地面数字电视广播模式的第三世界开始进行推广,已逐步完成产业化,价格大幅度下降,随后欧洲很多国家也开始采用,逐步替换DVB-T。
		\par DVB-T2亮点是能提供较高的系统净荷码率(最大净荷码率高达51Mbps,比目前有线及卫星数字电视能提供的单通道净荷码率还高),采用分集接收改进了单频网接收效果,接收门限更低等。\cite{ wiki:DVB-T}\cite{_what_is_dvb_t}
		\par 图\ref{fig:dvbt_tx}描述了DVB-T发射端系统。
		\begin{figure}[htb]
	\centering
	\begin{tikzpicture}[node distance=3em]
		\node(MPEG_2_stream)[process_vertical]{\rotatebox{-90}{MPEG-2 TS}\\流};
		\node(energy_dispersal)[process_vertical,right of=MPEG_2_stream]{能量扩散};
		\node(RS_encoder)[process_vertical,right of=energy_dispersal]{R\\S编码};
		\node(convolutional_interleaver)[process_vertical,right of=RS_encoder]{卷积内交织};
		\node(inner_coder)[process_vertical,right of=convolutional_interleaver]{内编码};
		\node(bit_inner_interleaver)[process_vertical,right of=inner_coder]{比特内交织};
		\node(symbol_inner_interleaver)[process_vertical,right of=bit_inner_interleaver]{符号内交织};
		\node(dvbt_map)[process_vertical,below=2em of symbol_inner_interleaver]{星座映射};
		\node(reference_signal)[process_vertical,left of=dvbt_map]{参考信号};
		\node(IFFT)[process_vertical,left of=reference_signal]{反傅里叶变换};
		\node(OFDM_CP)[process_vertical,left of=IFFT]{\rotatebox{-90}{OFDM}\\循环前缀};
		\node(rational_resampler)[process_vertical,left of=OFDM_CP]{重采样};
		\node(USRP_sink)[process_vertical,left of=rational_resampler]{\rotatebox{-90}{USRP}\\发射};
				
		\draw[arrow](MPEG_2_stream)--(energy_dispersal);
		\draw[arrow](energy_dispersal)--(RS_encoder);
		\draw[arrow](RS_encoder)--(convolutional_interleaver);
		\draw[arrow](convolutional_interleaver)--(inner_coder);
		\draw[arrow](inner_coder)--(bit_inner_interleaver);
		\draw[arrow](bit_inner_interleaver)--(symbol_inner_interleaver);
		\draw[arrow](symbol_inner_interleaver)--(dvbt_map);
		\draw[arrow](dvbt_map)--(reference_signal);
		\draw[arrow](reference_signal)--(IFFT);
		\draw[arrow](IFFT)--(OFDM_CP);
		\draw[arrow](OFDM_CP)--(rational_resampler);
		\draw[arrow](rational_resampler)--(USRP_sink);
	\end{tikzpicture}
	\caption{DVB-T发射端系统框图}
	\label{fig:dvbt_tx}
\end{figure}
\endinput
	\section{能量扩散}
		\par 当发射的信号过于集中在某一频段上,将对与其共用频段的其他系统或设备造成较大干扰。人为地对发射信号进行随机化或者加扰,可以使原本集中的信号能量均匀分布。同时,由于添加的随机码为伪随机码,所以在接收端很容易进行去随机化。在数字通信过程中,通过添加伪随机信号还能缩短连续的“0”或者连续的“1”的长度,接收端提取定时信号将变得更加容易。
	\section{RS编码}
	\section{卷积交织}
	\section{内编码(卷积编码)}
	\section{比特交织}
	\section{符号交织}
	\section{星座映射}
	\section{参考信号}
	\section{IFFT}
	\section{OFDM循环前缀}
	\section{常数}
	\section{重采样}
		\par 信号的采样率,用于满足另一个系统的要求
	\section{USRP发射}
