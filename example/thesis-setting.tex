%------------------------------------------------------------------------------%
%                                                                              %
%   LaTeX Template for Bachlor Thesis of Northwestern Polytechnical University %
%   Using XeLeTeX + MakeIndex + BibTeX, or Using CTeX v2.9.2.164               %
%   Version: 1.0.0                                                             %
%                                                                              %
%------------------------------------------------------------------------------%
%   Copyright 2016 by Shangkun Shen, MIT-LICENSE(see mit-license.polossk.com)  %
%------------------------------------------------------------------------------%


%---------------------------------纸张大小设置---------------------------------%
\usepackage{geometry}
    % 普通A4格式缩进
    % \geometry{left=2.5cm,right=2.5cm,top=2.5cm,bottom=2.5cm}
    % 论文标准缩进
    \geometry{left=1.25in,right=1.25in,top=1in,bottom=1.5in}
%------------------------------------------------------------------------------%


%----------------------------------必要库支持----------------------------------%
\usepackage{amsmath}
\usepackage{amssymb}
\usepackage{amsfonts}
\usepackage{mathrsfs}
\usepackage{bm}
\usepackage{xcolor}
\usepackage{tikz}
\usepackage{layouts}
\usepackage[numbers,sort&compress]{natbib}
\usepackage{clrscode}

\usepackage{dirtree}
\usepackage{url}
%------------------------------------------------------------------------------%


%--------------------------------设置标题与目录--------------------------------%
\usepackage[sf]{titlesec}
\usepackage{titletoc}
%------------------------------------------------------------------------------%


%--------------------------------添加书签超链接--------------------------------%
\usepackage[unicode=true,colorlinks=false,pdfborder={0 0 0}]{hyperref}
    % 在此处修改打开文件操作
    \hypersetup{
        bookmarks=true,         % show bookmarks bar?
        pdftoolbar=true,        % show Acrobat’s toolbar?
        pdfmenubar=true,        % show Acrobat’s menu?
        pdffitwindow=true,      % window fit to page when opened
        pdfstartview={FitH},    % fits the width of the page to the window
        pdfnewwindow=true,      % links in new PDF window
    }
    % 在此处添加文章基础信息
    \hypersetup{
        pdftitle={title},
        pdfauthor={author},
        pdfsubject={subject},
        pdfcreator={creator},
        pdfproducer={producer},
        pdfkeywords={key1  key2  key3}
    }
%------------------------------------------------------------------------------%


%---------------------------------设置字体大小---------------------------------%
\usepackage{type1cm}
% 字号与行距,统一前缀s(a.k.a size)
\newcommand{\sChuhao}{\fontsize{42pt}{63pt}\selectfont}         % 初号, 1.5倍
\newcommand{\sYihao}{\fontsize{26pt}{36pt}\selectfont}          % 一号, 1.4倍
\newcommand{\sErhao}{\fontsize{22pt}{28pt}\selectfont}          % 二号, 1.25倍
\newcommand{\sXiaoer}{\fontsize{18pt}{18pt}\selectfont}         % 小二, 单倍
\newcommand{\sSanhao}{\fontsize{16pt}{24pt}\selectfont}         % 三号, 1.5倍
\newcommand{\sXiaosan}{\fontsize{15pt}{22pt}\selectfont}        % 小三, 1.5倍
\newcommand{\sSihao}{\fontsize{14pt}{21pt}\selectfont}          % 四号, 1.5倍
\newcommand{\sHalfXiaosi}{\fontsize{13pt}{19.5pt}\selectfont}   % 半小四, 1.5倍
\newcommand{\sXiaosi}{\fontsize{12pt}{14.4pt}\selectfont}       % 小四, 1.25倍
\newcommand{\sLargeWuhao}{\fontsize{11pt}{11pt}\selectfont}     % 大五, 单倍
\newcommand{\sWuhao}{\fontsize{10.5pt}{10.5pt}\selectfont}      % 五号, 单倍
\newcommand{\sXiaowu}{\fontsize{9pt}{9pt}\selectfont}           % 小五, 单倍
%------------------------------------------------------------------------------%


%---------------------------------设置中文字体---------------------------------%
\usepackage{fontspec}
\usepackage[SlantFont,BoldFont,CJKchecksingle,CJKnumber]{xeCJK}
% 使用 Adobe 字体
\newcommand\adobeSog{Adobe Song Std}
\newcommand\adobeHei{Adobe Heiti Std}
\newcommand\adobeKai{Adobe Kaiti Std}
\newcommand\adobeFag{Adobe Fangsong Std}
\newcommand\codeFont{Consolas}
% 设置字体
\defaultfontfeatures{Mapping=tex-text}
\setCJKmainfont[ItalicFont=\adobeKai, BoldFont=\adobeHei]{\adobeSog}
\setCJKsansfont[ItalicFont=\adobeKai, BoldFont=\adobeHei]{\adobeSog}
\setCJKmonofont{\codeFont}
\setmonofont{\codeFont}
% 设置字体族
\setCJKfamilyfont{song}{\adobeSog}      % 宋体  
\setCJKfamilyfont{hei}{\adobeHei}       % 黑体  
\setCJKfamilyfont{kai}{\adobeKai}       % 楷体  
\setCJKfamilyfont{fang}{\adobeFag}      % 仿宋体
% 用于页眉学校名,特殊字体,powerby https://github.com/ecomfe/fonteditor
\setCJKfamilyfont{nwpu}{nwpuname}
% 新建字体命令,统一前缀f(a.k.a font)
\newcommand{\fSong}{\CJKfamily{song}}
\newcommand{\fHei}{\CJKfamily{hei}}
\newcommand{\fFang}{\CJKfamily{fang}}
\newcommand{\fKai}{\CJKfamily{kai}}
\newcommand{\fNWPU}{\CJKfamily{nwpu}}
%------------------------------------------------------------------------------%


%------------------------------添加插图与表格控制------------------------------%
\usepackage{graphicx}
\usepackage[font=small,labelsep=quad]{caption}
\usepackage{wrapfig}
\usepackage{multirow,makecell}
\usepackage{longtable}
\usepackage{booktabs}
\usepackage{tabularx}
\usepackage{setspace}
%------------------------------------------------------------------------------%


%---------------------------------添加列表控制---------------------------------%
\usepackage{enumerate}
\usepackage{enumitem}
%------------------------------------------------------------------------------%


%---------------------------------设置引用格式---------------------------------%
\renewcommand\figureautorefname{图}
\renewcommand\tableautorefname{表}
\renewcommand\equationautorefname{式}
\newcommand\myreference[1]{[\ref{#1}]}
\newcommand\eqrefe[1]{式(\ref{#1})}
\renewcommand\theequation{\thechapter.\arabic{equation}}
% 增加 \ucite 命令使显示的引用为上标形式
\newcommand{\ucite}[1]{$^{\mbox{\scriptsize \cite{#1}}}$}
%------------------------------------------------------------------------------%


%--------------------------------设置定理类环境--------------------------------%
\usepackage[amsmath,thmmarks]{ntheorem}
\newtheorem{myexample}{例}
\newtheorem{thm}{定理}
%------------------------------------------------------------------------------%


%--------------------------设置中文段落缩进与正文版式--------------------------%
\XeTeXlinebreaklocale "zh"       %使用中文的换行风格
\XeTeXlinebreakskip = 0pt plus 1pt    %调整换行逻辑的弹性大小
\usepackage{CJKnumb}
% \xeCJKcaption{gb_452}
\usepackage{indentfirst}
\setlength{\parindent}{2.0em}
\renewcommand\contentsname{目~~~~录}
\renewcommand\chaptername{\CJKprechaptername\CJKthechapter\CJKchaptername}
\setlength{\parskip}{3pt plus1pt minus1pt} % 段落间距
\renewcommand{\baselinestretch}{1.25} % 行距
%------------------------------------------------------------------------------%


%----------------------------设置段落标题与目录格式----------------------------%
\setcounter{secnumdepth}{4}
\setcounter{tocdepth}{4}


% 正文中标题格式,毋需标号
% \titleformat{\section}[hang]{\fHei \sf \sSihao}
%     {\sSihao }{0.5em}{}{}
% \titleformat{\subsection}[hang]{\fHei \sf \sHalfXiaosi}
%     {\sHalfXiaosi }{0.5em}{}{}
% \titleformat{\subsubsection}[hang]{\fHei \sf}
%     {\thesubsubsection }{0.5em}{}{}
% 正文中标题格式,需要标号

\renewcommand{\chaptername}{第\CJKnumber{\thechapter}章}
\renewcommand{\figurename}{图}
\renewcommand{\tablename}{表}
\renewcommand{\bibname}{参考文献}
\renewcommand{\contentsname}{目~录}
\newcommand{\keywords}[1]{\\ \\ \textbf{关~键~词}:#1}

\titleformat{\chapter}[hang]{\normalfont\sSanhao\filcenter\fHei\bf}
    {\sSanhao{\chaptertitlename}}{20pt}{\sSanhao}
\titleformat{\section}[hang]{\fHei \bf \sSihao}
    {\sSihao \thesection}{0.5em}{}{}
\titleformat{\subsection}[hang]{\fHei \bf \sHalfXiaosi}
    {\sHalfXiaosi \thesubsection}{0.5em}{}{}
\titleformat{\subsubsection}[hang]{\fHei \bf}
    {\thesubsubsection }{0.5em}{}{}
% 目录格式
\titlespacing{\chapter}{0pt}{-3ex  plus .1ex minus .2ex}{0.25em}
\titlespacing{\section}{0pt}{-0.2em}{0em}
\titlespacing{\subsection}{0pt}{-0.25em}{0em}
\titlespacing{\subsubsection}{0pt}{0.25em}{0pt}
% 缩小目录中各级标题之间的缩进
\dottedcontents{chapter}[0.0em]{\fHei\vspace{0.5em}}{0.9em}{5pt}
\dottedcontents{section}[1.16cm]{}{1.8em}{5pt}
\dottedcontents{subsection}[2.00cm]{}{2.7em}{5pt}
\dottedcontents{subsubsection}[2.86cm]{}{3.4em}{5pt}
%------------------------------------------------------------------------------%

%---------------------------------设置页眉页脚---------------------------------%
\usepackage{fancyhdr}
\usepackage{fancyref}
%\addtolength{\headsep}{-0.1cm}          %页眉位置
%\addtolength{\footskip}{-0.1cm}         %页脚位置
\addtolength{\topmargin}{0.5cm}
\newcommand{\makeheadrule}{
    \makebox[0pt][l]{\rule[.7\baselineskip]{\headwidth}{0.8pt}}
    \vskip-.8\baselineskip
}
\makeatletter
\renewcommand{\headrule}{%
    {
        \if@fancyplain\let\headrulewidth\plainheadrulewidth\fi
        \makeheadrule
    }
}
\pagestyle{fancyplain}
\fancyhf{}
\fancyfoot[C,C]{\sWuhao-~\thepage~-}
% 后续文字可以自行修改
\chead{\sSanhao\raisebox{0.04cm}{\fNWPU 西北工业大学} \fSong \bfseries{本科毕业设计论文}}
%------------------------------------------------------------------------------%


%-------------------------------数学符号格式控制-------------------------------%
\usepackage{bm}
\def\mathbi#1{\textbf{\em #1}}
% Caligraphic letters:      \mathcal{A}
% Mathbb letters:           \mathbb{A}
% Mathfrak letters:         \mathfrak{A}
% Math Sans serif letters:  \mathsf{A}
% MAth bold letters:        \mathbf{A}
% Math bold italic letters: \mathbi{A}
%------------------------------------------------------------------------------%


%-------------------------------数学特殊符号控制-------------------------------%
\newcommand\mi{{\mathrm i}}             % constant i
\newcommand\me{{\mathrm e}}             % constant e
\newcommand\mreal{{\mathbb R}}          % constant real set R
\newcommand\mhilb{{\mathbb H}}          % constant hilbert set H
\newcommand\mcond{{\mathrm{Cond.}}}     % condition symbol
\newcommand\mconst{{\mathrm{const}}}    % constant symbol
\newcommand\mve{{\bm e}}                % vector e
\newcommand\mvu{{\bm u}}                % vector u
\newcommand\mvv{{\bm v}}                % vector v
\newcommand\mvw{{\bm w}}                % vector w
\newcommand\mvx{{\bm x}}                % vector x
\newcommand\mvy{{\bm y}}                % vector y
\newcommand\mvz{{\bm z}}                % vector z
\newcommand\mvzero{{\bm 0}}             % vector 0
\newcommand\mvone{{\bm 1}}              % vector 1
\newcommand\mvalpha{{\bm \alpha}}       % vector alpha
\newcommand\mvomega{{\bm \omega}}       % vector omega
\newcommand\mma{{\mathbf A}}            % matrix A
\newcommand\mmb{{\mathbf B}}            % matrix B
\newcommand\mmd{{\mathbf D}}            % matrix D
\newcommand\mmh{{\mathbf H}}            % matrix H
\newcommand\mmi{{\mathbf I}}            % matrix I
\newcommand\mmk{{\mathbf K}}            % matrix K
\newcommand\mml{{\mathbf L}}            % matrix L
\newcommand\mmo{{\mathbf O}}            % matrix O
\newcommand\mmp{{\mathbf P}}            % matrix P
\newcommand\mms{{\mathbf S}}            % matrix S
\newcommand\mmt{{\mathbf T}}            % matrix T
\newcommand\mmtt{{{\mathbf T}^T}}       % matrix T^T
\newcommand\mmu{{\mathbf U}}            % matrix U
\newcommand\mmv{{\mathbf V}}            % matrix V
\newcommand\mmw{{\mathbf W}}            % matrix W
\newcommand\mmx{{\mathbf X}}            % matrix X
\newcommand\mmy{{\mathbf Y}}            % matrix Y
\newcommand\mmz{{\mathbf Z}}            % matrix Z
\newcommand\mmlambda{{\bm \Lambda}}     % matrix \Lambda
\DeclareMathOperator\diff{d\!}          % operator diff
\DeclareMathOperator\Diff{D\!}          % operator Diff
\DeclareMathOperator\Expect{E}          % operator Expect
\DeclareMathOperator\diag{diag}         % operator diag
\DeclareMathOperator\eig{eig}           % operator eig
\DeclareMathOperator\lcm{lcm}           % operator lcm
\DeclareMathOperator\tr{tr}             % operator tr
\DeclareMathOperator\var{var}           % operator var
\DeclareMathOperator\cov{cov}           % operator cov
\DeclareMathOperator\mean{mean}         % operator mean
\DeclareMathOperator*{\argmin}{argmin}  % operator argmin
\DeclareMathOperator{\dist}{dist}       % operator dist
\allowdisplaybreaks[4]
%------------------------------------------------------------------------------%

%----------------------------------其他补充设置--------------------------------%
% 重置列表环境的间隔
% \let\orig@Itemize =\itemize
% \let\orig@Enumerate =\enumerate
% \let\orig@Description =\description

% \def\Myspacing{
%     \itemsep=1.5ex \topsep=-0.5ex \partopsep=0pt \parskip=0pt \parsep=0.5ex
% }

% \def\newitemsep{
%     \renewenvironment{itemize}{\orig@Itemize\Myspacing}{\endlist}
%     \renewenvironment{enumerate}{\orig@Enumerate\Myspacing}{\endlist}
%     \renewenvironment{description}{\orig@Description\Myspacing}{\endlist}
% }

% \def\olditemsep{
%     \renewenvironment{itemize}{\orig@Itemize}{\endlist}
%     \renewenvironment{enumerate}{\orig@Enumerate}{\endlist}
%     \renewenvironment{description}{\orig@Description}{\endlist}
% }

% \newitemsep
% 下划线
\newcommand\dlmu@underline[2][5cm]{\hskip1pt\underline{\hb@xt@ #1{\hss#2\hss}}\hskip3pt}
\let\coverunderline\dlmu@underline
%------------------------------------------------------------------------------%



%----------------------------------添加代码控制--------------------------------%
\usepackage{listings}
\lstset{
    basicstyle=\footnotesize\ttfamily,
    %numbers=left,
    %numberstyle=\tiny,
    %numbersep=5pt,
    tabsize=4,
    extendedchars=true,
    breaklines=true,
    keywordstyle=\color{blue},
    numberstyle=\color{purple},
    commentstyle=\color{olive},
    stringstyle=\color{orange}\ttfamily,
    showspaces=false,
    showtabs=false,
    framexrightmargin=5pt,
    framexbottommargin=4pt,
    showstringspaces=false
    escapeinside=`', %逃逸字符(1左面的键),用于显示中文
}
\renewcommand{\lstlistingname}{CODE}
\lstloadlanguages{% Check Dokumentation for further languages, page 12
    Pascal, C++, Java, Ruby, Python, Matlab, R, sh
}
%------------------------------------------------------------------------------%

%----------------------------------一些常用的命令--------------------------------%
\newcommand{\tabincell}[2]{\begin{tabular}{@{}#1@{}}#2\end{tabular}}
%然后使用&\tabincell{c}{}&就可以在表格中自动换行

%------------------------------------------------------------------------------%
\endinput
% 这是简单的 thesis(article) 的导言区设置,不能单独编译。